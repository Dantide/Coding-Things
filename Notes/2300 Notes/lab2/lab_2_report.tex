\documentclass{article}
\usepackage[utf8]{inputenc}
\usepackage[a4paper,top=3cm,bottom=2cm,left=3cm,
right=3cm,marginparwidth=1.75cm]{geometry}
\usepackage{amsmath} %for \text inside math mode and matrices

\title{ECE 2300 - Lab 2 Report}
\author{Stephen Chin and Aryaa Pai}
\date{Due: February 20}


\begin{document}
\maketitle



\section{Introduction}

The overall goal of this lab was to sucessfully decode a four bit
input to an output number display on a seven section decoder. The lab
was split into two sections, both involving breadboarding inputs to
create a sucessful display. The first part allowed all inputs to be
sent through a CMOS CD4511B 7 segment decoder, allowing us to test our
initial breadboard setup, battery, and inputs. The second part allowed
us to wire the decoders for inputs $c$, $e$, and $g$ using only NAND
gates and inverters.

The materials allowed to us in this lab consisted of a breadboard, two
72LS00 2-input NAND gate chips, one 72LS04 NOT gate, one 72LS10
3-input NAND gate, one 72LS20 4-input NAND gate, one CMOS CD4511B 7-
segment decoder, one 74LS241 buffer, a resistor array, a 7-segment
input display, along with a battery and a voltage regulator.

During our time in the lab, we were able to complete our design,
achieving a fully functioning 7 segment display output on all digits 0
to 9. Although difficult to implement, we were able to take our
pre-designed circuit and wire it correctly on the breadboard.


\section{Design}

% Insert picture here

The design of our logic circuit was based on the process of logic
minimization through the use of Karnaugh maps. In order to create a
minimized logic function for each segment of the 7-segment display, we
created Karnaught maps for each of the outputs based on a truth table
created from looking at the desired outputs. By our design
specification, we were able to consider all inputs 10-15 to be ''don't
cares'', simplifying our logic expressions.



\section{Implementation and Testing}



\section{Conclusion}



\section{Work Distribution}



\end{document}

