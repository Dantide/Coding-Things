\documentclass{article}
\usepackage[utf8]{inputenc}
\usepackage[letterpaper,top=3cm,bottom=2cm,left=3cm,right=3cm,
marginparwidth=1.75cm]{geometry}
\usepackage{amsmath}
\usepackage{amssymb}
\usepackage{enumitem}


\title{ECE 3100 - PSet 3}
\author{Stephen Chin}
\date{February 20}


\begin{document}
\maketitle

\section*{Answers}

\begin{enumerate}[itemsep=12pt]
\item
  Show that $\mathbb{P} (A_1^c \cup A_2^c \cup \dots \cup A_n^c) =
  1 - \mathbb{P}(A_1) \mathbb{P}(A_2) \dots \mathbb{P}(A_n)$.

  Well, if we take the complement of $A_1^c \cup A_2^c \cup \dots \cup A_n^c$, then we are left with
  the set of all outcomes that are not in the complement of any of the events $A_1$ to $A_n$. Thus,
  we are left with $A_1 \cap A_2 \cap \dots \cap A_n$. By definition of independence, we can rewrite
  this as a product of all the events $A_1$ through $A_n$, bringing us to our goal.
  
  \[
    \begin{aligned}
      \mathbb{P} (A_1^c \cup A_2^c \cup \dots \cup A_n^c)
      &= 1 - \mathbb{P} ((A_1^c \cup A_2^c \cup \dots \cup A_n^c)^c) \\
      &= 1 - \mathbb{P} (A_1 \cap A_2 \cap \dots \cap A_n) \\
      &= 1 - \mathbb{P} (A_1) \mathbb{P} (A_2) \dots \mathbb{P} (A_n)
    \end{aligned}
  \]
  

\item
  Rangers and Canadiens

  Let us define the events $A = { \text{event that the Canadiens win}}$ and $B = { \text{the event
      that the rangers win}}$. We know that there can only be one winner, so $\mathbb{P} ( A \cap B
  ) = 0$.
  
  Not enough time to complete question.


\item
 Communication System Transmissions

  \begin{enumerate}
  \item
    Let $A_i = { \text{the event that the i-th transistor sends a message}}$.
    $\mathbb{P} (\text{a successful transmission}) = p(1 - p)^{n-1}$.

  \item
    The derivative of $p(1 - p)^{n-1}$ is $(1 - p)^{(-2 + n)} (1 - n p)$, which has zeros at the
    values 1 and $1/n$. Because the range of possible values of p is $[0, 1]$, we also check the
    value of p at 0. At $p = 1$ and $p = 0$, the probability of a successful transmission is $0$. At
    $p = 1/n$, the probability of a successful trasmission is
    $\frac{1}{n} (1 - \frac{1}{n})^{n - 1}$. This is the relative maximum for $p$ on the interval
    $[0, 1]$.

  \item
    The $\lim_{x \to \infty} \frac{1}{n} (1 - \frac{1}{n})^{n - 1} = 0$. This makes sense, because
    as we get more and more transmitters in our system, we are more likely to have two transmitters
    that send their transmissions at the same time. The more tranmitters we have, the less likely we
    send out a successful transmission.

  \item
    Not enough time to complete question.
  \end{enumerate}


\item
  Communication System w/ Time Slots

  \begin{enumerate}
  \item
    The probability that a successful transmission is sent within $k$ time slots is equal to one
    minus the probability that no successful transmissions are sent within $k$ time slots. Let $F =
    \{$the event that no successful transmissions are sent in the first $k$ time
      slots$\}$. $\mathbb{P} (F) = k (1 - p (1 - p)^{n - 1})$, because we would need to have no
    successful transmissions sent $k$ times.

  \item
    Not enough time to complete question.

  \item
    Not enough time to complete question.

  \item
    Not enough time to complete question.
  \end{enumerate}

\item
  Banking Password

  Not enough time to complete question.
  
  
\end{enumerate}



\end{document}