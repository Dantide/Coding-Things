\documentclass{article} \usepackage[utf8]{inputenc}
\usepackage[letterpaper,top=3cm,bottom=2cm,left=3cm,right=3cm,
marginpaperwidth=1.75cm]{geometry} \usepackage{amsmath}
\usepackage{amssymb} \usepackage{enumitem}


\title{ECE 3100 - Functions, Formulas, and Definitions}
\author{Stephen Chin} \date{Spring Semester 2019}


\begin{document}
\maketitle

\section{Pre - Prelim 1}

\subsection{Lecture 1 - What is Probability?}

Probability is a way of mathematically modelling situations involving
uncertainty with the goal of making predications decisions and models.
Probability can be understood in many ways, such as:

\begin{enumerate}
\item Frequency of Occurence: Or percentage of successes in a
  moderately large number of similar situations.
  
\item Subjective belief: Or ceratinty based on other understood facts
  about a claim.
\end{enumerate}

For our Probability Models, we define the set of all outcomes to be
$\Omega$, better known as the \textbf{sample space} of an
experiment. All subsets of $\Omega$ are called \textbf{events}. These
are both sets and can be understood using default set notation.


\subsection{Lecture 2 - Probability Law}

Given $\Omega$ chosen, a \textbf{probability law} on $\Omega$ is a
mapping $\mathbb{P}$ that assings a number for every event such that:

\begin{equation} \tag{Kolmogorov's Axioms} \boxed{
    \begin{aligned} \mathbb{P}(A) \ge 0 & \quad \text{for every event
        A} \\ \mathbb{P}(\Omega) = 1 & \quad \text{(normalization)}
    \end{aligned} }
\end{equation}

\underline{Additivity rules:}
\begin{itemize}
\item If $A \cap B = \varnothing$, ($A, B$) events, then:
  \begin{equation}
    \boxed{
      \mathbb{P}(A \cup B) = \mathbb{P}(A) + \mathbb{P}(B)
    }
  \end{equation}

\item If events $A_1, A_2, \dots$ are all disjoint, then:
  \begin{equation}
    \boxed{
      \mathbb{P} (\bigcup\limits_{n=1}^{\infty} A_n) =
      \sum_{n=1}^{\infty} \mathbb{P}(A_n)
    }
  \end{equation}
\end{itemize}

By these rules, we can surmise that $\boxed{\mathbb{P} (\varnothing) =
  0}$.

For any events $A, B$:
\begin{equation}
  \tag{Event Union}
  \boxed{
    \mathbb{P}(A \cup B) = \mathbb{P}(A) + \mathbb{P}(B) -
    \mathbb{P}(A \cap B)
  }
\end{equation}

When we have a probability law on a finite $\Omega$ with all outcomes
equally likely (i.e. $\mathbb{P}(\{s\}) = 1/size(\Omega)$), we call
this probability law $\mathbb{P}$ a \textbf{(discrete uniform
  probability law}.


\subsection{Lecture 3 - Conditional Prob \& Product Rule}

Conditional Probability is defined $\mathbb{P}(A \mid B)$ =
``Probability of A given B''. It is understood as the likelyhood that
event A occurs, given that B also occurs. 

\begin{equation}
  \tag{Conditional Probability Definition}
  \boxed{
    \mathbb{P}(A \mid B) = \frac{\mathbb{P}(A \cap B)}{\mathbb{P}(B)}
  }
\end{equation}


\end{document}